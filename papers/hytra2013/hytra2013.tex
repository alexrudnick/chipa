\documentclass[11pt]{article}
\usepackage{acl2013}
\usepackage{times}
\usepackage{url}
\usepackage{latexsym}
\usepackage{listings}
\usepackage{float}
\floatstyle{boxed}
\restylefloat{figure}
\lstset{
language=Python,
basicstyle=\small\sffamily,
numbers=none,
numberstyle=\tiny,
frame=tb,
columns=fullflexible,
showstringspaces=false
}
%\setlength\titlebox{6.5cm}    % You can expand the title box if you
% really have to

%% helpful suggestions from reviewers
% TODO please provide linguistic examples and intuitions for the approach
% - (maybe just the explanation with math is not super-helpful)

%% TODO clarify the situation with windows. what do we mean by a three-word
% window?

%% TODO are the results statistically significant?

% DONE clarify about what we mean by 't', because we use that to refer to both
% pos tags and target-language words
%% DONE "At the end of the first paragraph of section 7, we can read that the
% use of the sequence of previous labels is informative, since MaxEnt and MEMM
% outperform HMM systems. However, in both Figure 1 and Figure 2, all
% approaches include previous labels. In fact, the feature that seems to make
% the difference is the three-word window, that is present in MaxEnt and MEMM
% and it is not in HMM."
%% DONE Define MFS.
%% DONE make it more clear what we mean by "precision" and "accuracy"; the
% reviewer didn't know whether they were the same thing. (maybe they missed
% it?)


\title{Lexical Selection for Hybrid MT with Sequence Labeling}


\author{Alex Rudnick and Michael Gasser\\
        Indiana University, School of Informatics and Computing \\
        {\tt \{alexr,gasser\}@indiana.edu}}

\date{}

\begin{document}
\maketitle
\begin{abstract}
We present initial work on an inexpensive approach for building
large-vocabulary lexical selection modules for hybrid RBMT systems by framing
lexical selection as a sequence labeling problem. We submit that Maximum
Entropy Markov Models (MEMMs) are a sensible formalism for this problem, due to
their ability to take into account many features of the source text, and show
how we can build a combination MEMM/HMM system that allows MT system
implementors flexibility regarding which words have their lexical choices
modeled with classifiers. We present initial results showing successful use of
this system both in translating English to Spanish and Spanish to Guarani.
\end{abstract}

\section{Introduction}
Lexical ambiguity presents a serious challenge for rule-based machine translation
(RBMT) systems, since
many words have several possible translations in a given target language, and
more than one of them may be syntactically valid in context. A translation
system must choose a translation for each word or phrase in the input sentence,
and simply taking the most common translation will often fail, as a word in the
source language may have translations in the target language with significantly
different meanings. Even when choosing among near-synonyms, we would like to
respect selectional preferences and common collocations to produce
natural-sounding output text.

Writing lexical selection rules by hand is tedious and error-prone; even if
informants familiar with both languages are available, they may not be able to
enumerate the contexts under which they would choose one translation
alternative over another. Thus we would like to learn from corpora where
possible. 

Framing the resolution of lexical ambiguities in machine translation
as an explicit classification
task has a long history, dating back at least to early SMT work at IBM
\cite{Brown91word-sensedisambiguation}.  More recently, Carpuat and Wu have
shown how to use word-sense disambiguation techniques to improve modern
phrase-based SMT systems \cite{carpuatpsd}, even though the language model and
phrase tables of these systems can mitigate the problem of lexical ambiguities
somewhat. Treating lexical selection as a word-sense disambiguation problem, in
which the sense inventory for each source-language word is its set of possible
translations, is often called cross-lingual WSD (CL-WSD). This framing has
received enough attention to warrant shared tasks at recent SemEval workshops;
the most recent running of the task is described in \cite{task10}.

Intuitively, machine translation implies an ``all-words" WSD task: we need to
choose a translation for every word or phrase in the source sentence, and the
sequence of translations should make sense taken together. Here we begin to
explore CL-WSD not just as a classification task, but as one of sequence
labeling. We describe our approach and implementation, and present two
experiments. In the first experiment, we apply the system to the SemEval 2013
shared task on CL-WSD \cite{task10}, translating from English to Spanish, and
in the second, we perform an all-words labeling task, translating text from the
Bible from Spanish to Guarani. This is work in progress and our code is
currently ``research-quality", but we are developing the software in the
open\footnote{\url{http://github.com/alexrudnick/clwsd}}, with the intention of
using it with free RBMT systems and producing an easily reusable package as the
system matures.

\section{Related Work}
To our knowledge, there has not been work specifically on sequence labeling
applied to lexical selection for RBMT systems. However, 
there has been work recently on using WSD techniques for translation into
lower-resourced languages, such as the English-Slovene language pair, as in 
\cite{vintar-fivser-vrvsvcaj:2012:ESIRMT-HyTra2012}. 

The Apertium team has a particular practical interest in improving lexical
selection in RBMT; they recently have been developing
a new system, described in \cite{tyers-fst}, that learns finite-state
transducers for lexical selection from the available parallel corpora. It is
intended to be both very fast, for use in practical translation systems, and
to produce lexical selection rules that are understandable and modifiable by
humans.

Outside of the CL-WSD setting, there has been work on framing all-words WSD as
a sequence labeling problem. Particularly, Molina \textit{et al.}
\shortcite{DBLP:conf/iberamia/MolinaPS02} have made use of HMMs for all-words
WSD in a monolingual setting.

\section{Sequence Labeling with HMMs}
In building a sequence-based CL-WSD system, we first tried using the familiar
HMM formalism. An HMM is a generative model, giving us a formula for $P(S, T) =
P(T) * P(S|T)$. Here by $S$ we mean a sequence of source-language words, and by
$T$ we mean a sequence words or phrases in the target language. In practice,
the input sequence $S$ is a given, and we want to find the sequence $T$ that
maximizes the joint probability, which means predicting an appropriate label
for each word in the input sequence.

Using the (first-order) Markov assumption, we approximate $P(T)$ as $P(T) =
\prod_{i} P(t_i | t_{i-1})$, where $i$ denotes each index in the input
sentence. Then we imagine that each source-language word $s_i$ is generated by
the corresponding unobserved label $t_i$, through the emission probabilities
$P(s|t)$. This generative model is admittedly less intuitive for CL-WSD than
for POS-tagging (where it is more traditionally applied), in that it requires
the target-language words to be generated in the source order.

Training the transition model -- roughly an n-gram language model -- for
target-language words or phrases in the source order is straightforward with
sentence-aligned bitext. We use one-to-many alignments in which each source
word corresponds with zero or more target-language words, and we take the
sequence of target-language words aligned with a given source word to be its
label. NULL labels are common; if a source word is not aligned to a target
word, it gets a NULL label. Similarly , we can learn the emission
probabilities, $P(s|t)$, simply by counting which source words are paired with
which target words and smoothing.

For decoding with this model, we can use the Viterbi algorithm, especially for
a first-order Markov model -- although we must be careful in the inner loops
only to consider the possible target-language words and not the entire
target-language vocabulary. The Viterbi algorithm may still be used with
second- or higher-order models, although it slows down considerably. In the
interest of speed, in this work we performed decoding for second-order HMMs
with a beam search.

\section{Sequence Labeling With MEMMs and HMMs}
Contrastingly, an MEMM is a discriminative sequence model, with
which we can calculate the conditional probability $P(T|S)$ using individual
discriminative classifiers that model $P(t_i | F)$ (for some features $F$).
Like an HMM, an MEMM models transitions over labels, although the
input sequence is considered given. This frees us to include any features we
like from the source-language sentence. The ``Markov" aspect of the MEMM is
that, unlike a standard maximum entropy classifier, we can include information
from the previous $k$ labels as features, for a $k$-th order MEMM. So at every
step in the sequence labeling, we want a classifier that models 
$P(t_i | S, t_{i-1}...t_{i-k})$, and the probability of a sequence $T$ is just
the product of each of the individual transition probabilities.

To avoid the intractable task of building a single classifier that might return
thousands of different labels, we could in principle build a classifier for
each individual word in the source-language vocabulary, each of which will
produce perhaps tens of possible target-language labels. However, there will be
tens or hundreds of thousands of words in the source-language vocabulary, and
most word-types will only occur very rarely; it may be prohibitively expensive
to train and store classifiers for each of them.

We would like a way to focus our efforts on some words, but not all, and to
back off to a simpler model when a classifier is not available for a given
word. Here, in order to approximate $P(t_i | S, t_{i-1}...t_{i-k})$,
we use an HMM, as described in the previous section, with which we can estimate
$P(s_i, t_i | t_{i-1}...t_{i-k})$ as  
$P(t_i | t_{i-1}...t_{i-k}) * P(s_i | t_i)$.
This gives us the joint probability, which we divide by $P(s_i)$
-- prior probabilities of each source-language word must be stored ahead of
time -- and thus we can approximate the conditional probability that we need to
continue the sequence labeling.

In the implementation, we can specify criteria under which a source-language
word will have its translations explicitly modeled with a maximum entropy
classifier. When training a system, one might choose, for example, the 100 most
common ambiguous words, all words that are observed a certain number of times
in the training corpus, or words that are particularly of interest for some
other reason.

At training time, we find all of the instances of the words that we want to
model with classifiers, along with their contexts, so that we can extract
appropriate features for training the classifiers. Then we train classifiers
for those words, and store the classifiers in a database for retrieval at
inference time.

For inference with this model, we use a beam search rather than the Viterbi
algorithm, for convenience and speed while using a second-order Markov model.
A sketch of the beam search implementation is presented in Figure
\ref{fig:beamsearch}.

\begin{figure*}
\begin{lstlisting}[frame=none]
def beam_search(sequence, HMM, source_word_priors, classifiers):
    """Search over possible label sequences, return the best one we find."""
    candidates = [Candidate([], 0)] # empty label sequence with 0 penalty
    for t in range(len(sequence)):
        sourceword = sequence[t]
        for candidate in candidates:
            context = candidate.get_context(t) # labels at positions (t-2, t-1)
            if sourceword in classifiers:
                features = extract_features(sequence, t, context)
                label_distribution = classifiers[sourceword].prob_classify(features)
            else:
                label_distribution = Distribution()
                for label in get_vocabulary(sourceword):
                    label_distribution[label] = (HMM.transition(context, label) +
                                                 HMM.emission(sourceword, label) -
                                                 source_word_priors[sourceword])
            # extend candidates for next time step to include labels for next word
            add_new_candidates(candidate, label_distribution, new_candidates)
        candidates = filter_top_k(new_candidates, BEAMWIDTH)
    return get_best(candidates)
\end{lstlisting}
\caption{Python-style code sketch for MEMM/HMM beam search. Here we are using
negative log-probabilities, which we interpret as penalties to be minimized.}
\label{fig:beamsearch}
\end{figure*}

\section{Experiments}
So far, we have evaluated our sequence-labeling system in two different
settings, the English-Spanish subset of a recent SemEval shared task
\cite{task10},
and an all-words prediction task in which we want to translate, from Spanish to
Guarani, each word in a test set sampled from the Bible.

\subsection{SemEval CL-WSD task}
In the SemEval CL-WSD task, systems must provide translations for twenty
ambiguous English nouns, given a small amount of context, typically a single
sentence. The test set for this task consists of fifty short passages for each
ambiguous word, for a thousand test instances total; each passage contains one
or a few uses of the ambiguous word. For each test passage, the system must
produce a translation of the noun of interest into the target language.  These
translations may be a single word or a short phrase in the target language, and
they should be lemmatized. The task allows systems to produce several output
labels, although the scoring metric encourages producing one best guess, which
is matched against several reference translations provided by human annotators.
The details of the scoring are provided in the task description paper, and the
scores reported were calculated with a script provided by the task organizers.

For simplicity and comparability with previous work, we trained our system on
the Europarl Intersection corpus, which was provided for developing CL-WSD
systems in the shared task.  The Europarl Intersection is a subset of the
sentences from Europarl \cite{europarl} that are available in English and all
five of the target languages for the task, although for these initial
experiments, we only worked with Spanish. There were 884603 sentences in our
training corpus.

We preprocess the Europarl training data by tokenizing with the default NLTK
tokenizer \cite{nltkbook}, getting part-of-speech tags for the English text
with the Stanford Tagger \cite{Toutanova03feature-richpart-of-speech}, and
lemmatizing both sides with TreeTagger \cite{Schmid95improvementsin}.  We
aligned the untagged English text with the Spanish text using the Berkeley
Aligner \cite{denero-klein:2007:ACLMain} to get one-to-many alignments from
English to Spanish, since the target-language labels in this setting may be
multi-word phrases. We used nearly the default settings for Berkeley Aligner,
except that we ran 20 iterations each of IBM Model 1 and HMM alignment.

We trained classifiers for all of the test words, and also for any words that
appear more than 500 times in the corpus. The classifiers used the previous two
labels and all of the tagged, lemmatized words within three words on either
side of the target word as features. Training was done with the MEGA Model
optimization package \footnote{\url{http://www.umiacs.umd.edu/~hal/megam/}} and
its corresponding NLTK interface.

At testing time, for each test instance, we labeled the test sentences with
four different sequence labeling methods: first-order HMMs, second-order HMMs,
MaxEnt classifiers with no sequence features, and the MEMMs with HMM backoff.
We then compared the system output against the reference translations for the
target words using the script provided by the task organizers.

\subsection{All-words Lexical Selection for Spanish-Guarani}
Since we are primarily interested in lexical selection for RBMT systems in
lower-resource settings, we also experimented with translating from Spanish to
Guarani, using the Bible as bitext. In this experiment, we labeled all of the
text in the test set using each of the different sequence labeling models, and
we report the classification accuracy over the test set.

In preparing the corpus, since different translations of the Bible do not
necessarily have direct correspondences between verse numbers (they are not
unique identifiers across language!), we selected only the chapters that contain
the same number of verses in our Spanish and Guarani translations.  This only
leaves 879 chapters, out of 1189 total, for a total of 22828 bitext verses of
roughly one sentence each. We randomly sampled 100 verses from the corpus and
set these aside as the test set.

Here we trained the HMM and MEMM as before, but with lemmatized Spanish as the
source language, and the roots of Guarani words as the target.  As Guarani is
much more morphologically rich language than either English or Spanish, this
requires the use of a sophisticated morphological analyzer, which is described
in section \ref{sec:guaranima}. Due to the much smaller data set, in this
setting we stored classifiers for any Spanish word that occurs more than 20
times in the training data and backed off to the HMM during decoding otherwise.

\section{Morphological Analysis for Guarani}
\label{sec:guaranima}
We analyze the Spanish and Guarani Bible using our in-house morphological
analyzer, originally developed for Ethiopian Semitic languages 
\cite{gasser:eacl09}.
As in other, more familiar, modern
morphological analyzers such as \cite{beesley+karttunen}, analysis in our
system is modeled by cascades of finite-state transducers (FSTs).  To solve the
problem of long-distance dependencies, we extend the basic FST framework using
an idea introduced by Amtrup \shortcite{amtrup:03}.  Amtrup starts with the
well-understood framework of weighted FSTs, familiar from speech recognition.
For speech recognition, FST arcs are weighted with probabilities, and a
successful traversal of a path through a transducer results in a probability
that is the product of the probabilities on the arcs that are traversed, as
well as an output string as in conventional transducers.  Amtrup showed that
probabilities could be replaced by feature structures and multiplication by
unification.  In an FST weighted with feature structures, the result of a
successful traversal is the unification of the feature structure ``weights'' on
the traversed arcs, as well as an output string.  Because a feature structure
is accumulated during the process of transduction, the transducer retains a
sort of memory of where it has been, permitting the incorporation of
long-distance constraints such as those relating the negative prefix and suffix
of Guarani verbs.

In our system, the output of the morphological analysis of a word is a root and
a feature structure representing the grammatical features of the word.  We
implemented separate FSTs for Spanish verbs, for Guarani nouns, and for the two
main categories of Guarani verbs and adjectives.  Since Spanish nouns and
adjectives have very few forms, we simply list the alternatives in the lexicon
for these categories.  For this paper, we are only concerned with the roots of
words in our corpora, so we ignore the grammatical features that are output
with each word.

\floatstyle{plain}
\restylefloat{figure}
\begin{figure*}[t!]
  \begin{center}
  \begin{tabular}{|r|l|r|}
    \hline
    system & features & score (precision) \\
    \hline
     MFS (with tag) &                                 & 24.97 \\
     MFS (without tag) &                              & 23.23 \\
    \hline
     HMM1    & current word, previous label           & 21.17 \\
     HMM2    & current word, previous two labels      & 21.23 \\
     MaxEnt  & three-word window                      & 25.64 \\
     MEMM    & three-word window, previous two labels & \textbf{26.49} \\
    \hline
  \end{tabular}
  \end{center}
\caption{Results for the first experiment; SemEval 2013 CL-WSD task.}
\label{fig:theresults}
\end{figure*}

\floatstyle{plain}
\restylefloat{figure}
\begin{figure*}[t!]
  \begin{center}
  \begin{tabular}{|r|l|r|}
    \hline
    system & features & score (accuracy \%) \\
    \hline
    MFS      &                                        & 60.39  \\
    \hline
     HMM1    & current word, previous label           & 57.40  \\
     HMM2    & current word, previous two labels      & 43.04  \\
     MEMM    & three-word window, previous two labels & \textbf{66.82}  \\
    \hline
  \end{tabular}
  \end{center}
\caption{Results for the second experiment; all-words lexical selection on the
Guarani Bible}
\label{fig:theresults2}
\end{figure*}

\section{Results}
The scores for the first experiment are presented in Figure
\ref{fig:theresults}. Here we use the precision metric calculated by the
scripts for the SemEval shared task \cite{task10}, which compare the answers
produced by the system against several reference answers given by human
annotators. There are two ``most-frequent sense" baselines reported.  The first
one (``with tag"), is the baseline in which we always take the most frequent
label for a given source word, conditioned on its POS tag. The other MFS
baseline is not conditioned on POS tag; this was the baseline for the SemEval
task.  Perhaps unsurprisingly, we see part-of-speech tagging doing some of the
lexical disambiguation work.

Neither of the HMM systems beat the most-frequent-sense baselines, but both the
non-sequence MaxEnt classifier and the MEMM system did, which suggests that the
window features are useful in selecting target-language words. Furthermore, the
MEMM system outperforms the MaxEnt classifiers.

The scores for the second experiment are presented in Figure
\ref{fig:theresults2}. Here we did not have human-annotated
reference translations for each word, so we take the labels extracted from the
alignments
as ground truth and can only report per-word classification accuracy,
rather than the more sophisticated precision metric used in the shared task.

Here we see similar results. Neither of the HMM systems beat the MFS baseline,
and the trigram model was noticeably worse. The training set here is probably
too sparse to train a good trigram model. The MEMM system, however, did beat
the baseline, posting the highest results: just over two-thirds of the time, we
were able to predict the correct label for each Spanish word, whereas the
most-frequent label was correct about 60\% of the time.

%%English to Spanish
%%unigrams mean: 24.9745
%%bigrams mean: 21.169999999999998
%%trigrams mean: 21.227
%%
%%now with classifiers
%%maxent mean: 25.64
%%memms mean: 26.49

%% Spanish to Guarani, all words accuracy
%% unigrams
%% accuracy: 0.603943661971831
%% bigrams
%% accuracy: 0.5740845070422536
%% trigrams
%% accuracy: 0.4304225352112676
%% MEMM: MEMM BEATS MOST FREQUENT SENSE
%% accuracy: 0.668169014084507

\section{Conclusions and Future Work}
We have described a work-in-progress lexical selection system that takes a
sequence labeling approach, and shown some initial successes in using it for
cross-language word sense disambiguation tasks for English to Spanish and
Spanish to Guarani.  We have demonstrated a hybrid sequence labeling strategy
that combines MEMMs and HMMs, which will allow users to set parameters sensibly
for their computational resources and available training data.

In future work, we will continue to refine the approach, exploring different
parameter settings, such as beam widths, numbers of classifiers for the MEMM
component, and the effects of different features as input to the classifiers.
We are also interested in making use of multilingual information sources,
as in the work of Lefever and Hoste
\shortcite{lefever-hoste-decock:2011:ACL-HLT2011}. We may also consider more
sophisticated sequence tagging models, such as CRFs
\cite{DBLP:conf/icml/LaffertyMP01}, although we may not have enough training
data to make use of richer models.

Our goal for this work is practical; we are trying to produce a hybrid
Spanish-Guarani MT system that can be used in Paraguay. We have a small amount
of Guarani training data available, and plan to collect more.  At the time of
writing, our lexical selection system is a prototype and not yet integrated
with our RBMT engine, but this integration is among our near-term goals.

A limitation of the current design is that we do not yet have a good way to
make use of monolingual training data. In SMT, it is common practice to train a
language model for the target language from a monolingual corpus that is much
larger than the available bitext. There is a substantial amount of available
Guarani text on the Web, but our current
model can only be trained on aligned bitext. Given Guarani text that had been
rearranged into a Spanish-like word order, we could build a better model for
the transition probabilities in the HMM component of the system. It might be
feasible to use a Guarani-language parser and some linguistic knowledge for
this purpose.
We will also investigate ways to translate multiword expressions as a unit
rather than word-by-word.

\bibliographystyle{acl.bst}
\bibliography{hytra2013.bib}{}

\end{document}
