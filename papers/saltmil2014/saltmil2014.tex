\documentclass[10pt, a4paper]{article}
\usepackage{lrec2006}
\usepackage{graphicx}
\usepackage{url}
\usepackage{latexsym}
\usepackage{listings}
\usepackage[utf8]{inputenc}

\title{Enhancing a Rule-Based MT System with Cross-Lingual WSD}

\name{Alex Rudnick$^1$, Annette Rios$^2$, Michael Gasser$^1$}

\address{$^1$ School of Informatics and Computing, Indiana University \\
         $^2$ Institute of Computational Linguistics, University of Zurich \\
         \texttt{\{alexr,gasser\}@indiana.edu}, \texttt{arios@ifi.uzh.ch} \\}

\abstract{
Lexical ambiguity is a significant problem facing rule-based machine
translation systems, as many words have several possible translations in a
given target language, each of which can be considered a sense of the word from
the source language.
The difficulty of resolving these ambiguities is mitigated for 
statistical machine translation systems for language pairs with large bilingual
corpora, as large n-gram language models and phrase tables containing common
multi-word expressions can encourage coherent word choices.
For most language pairs these resources are not available, so a primarily
rule-based approach becomes attractive.
In cases where some training data is available, though, we can
investigate hybrid RBMT and machine learning approaches, leveraging small and
potentially growing bilingual corpora. In this paper we
describe the integration of statistical cross-lingual word-sense disambiguation
software with SQUOIA, an existing rule-based MT system for the Spanish-Quechua
language pair, and show how it allows us to learn from the available bitext to
make better lexical choices, with very few code changes to the base system. We
also describe Chipa, the new open source CL-WSD software used for these
experiments.
\\ \newline
\Keywords{under-resourced languages, hybrid machine translation, word-sense
disambiguation}}

\begin{document}

\maketitleabstract

\section{Introduction}
Here we report on the development of Chipa, a package for statistical
lexical selection, and on integrating it into
SQUOIA\footnote{\url{http://code.google.com/p/squoia/}}, a primarily rule-based
machine translation system for the Spanish-Quechua language pair.  With very
few code changes to SQUOIA, we were able to make use of the lexical suggestions
provided by Chipa.

The integration enables SQUOIA to take advantage of any available bitext
without significantly changing its design, and to improve its word choices as
additional bitext becomes available. Our initial experiments also suggest that
we are able to use unsupervised approaches on monolingual Spanish text to
further improve results.

In this paper, we describe the designs of the Chipa and SQUOIA systems, discuss
the data sets used, and give results on both how well Chipa is able to learn
lexical selection classifiers in isolation, and to what extent it is able to
improve the output of SQUOIA on a full Spanish-to-Quechua translation task.

In its current design, SQUOIA makes word choices based on its bilingual
lexicon; the possible translations for a given word or multi-word expression
are retrieved from a dictionary on demand. If there are several possible
translations for a lexical item, these are passed along the pipeline in hopes
that later stages will make a decision, but if no disambiguation decision is
made, then the first entry retrieved from the lexicon is used. While there are
some rules for lexical selection, they have been written by hand and only cover
a small subset of the vocabulary, in a limited number of contexts.

In this work, we supplement these rules with classifiers learned from
Spanish-Quechua bitext. These classifiers make use of regularities that may not
be obvious to human rule-writers, providing improved lexical selections for
any word type with adequate coverage in the training corpus.

Quechua is a group of closely related indigenous American languages spoken in
South America. There are many dialects of Quechua; SQUOIA focuses on the
Cuzco dialect, spoken around the Peruvian city of Cuzco.  Cuzco Quechua has
several million speakers, and some useful linguistic resources available,
including a small treebank \cite{rios2009quechua}, also produced by the SQUOIA
team.

\section{SQUOIA}
SQUOIA is a shallow-transfer RBMT system based on the
architecture of MATXIN \cite{matxin_2005,matxin}.
The core system relies on a classical transfer approach and is mostly
rule-based, with a few components based on machine learning.
SQUOIA uses a pipeline approach, both in an abstract architectural sense and in
the sense that its pieces are instantiated as a series of scripts that communicate
via UNIX pipes. Each module performs some transformation on its input and
passes along the updated version to the next stage. Many modules focus on very
particular parts of the representation, leaving most of their input unchanged.

In the first stages, Spanish source sentences are analyzed with off-the-shelf
open-source NLP tools. To analyze the input Spanish text,
SQUOIA uses FreeLing \cite{padro12} for morphological analysis and named-entity
recognition,
Wapiti \cite{lavergne2010practical} for tagging,
and DeSr \cite{attardi-EtAl:2007:EMNLP-CoNLL2007} for parsing.
All of these modules rely on statistical models.

In the next step, the Spanish verbs must be disambiguated in order to assign
them a Quechua verb form for generation: a rule-based module tries to assign a
verb form to each verb chunk based on contextual information. If the rules fail to
do so due to parsing or tagging errors, the verb is marked as ambiguous and
passed on to an SVM classifier, which assigns a verb form even if the context
of that verb does not unambiguously select a target form. This is among the
most difficult parts of the
translation process, as the grammatical categories encoded in verbs differ
substantially between Spanish and Quechua. In the next step, a lexical transfer
module inserts all possible translations for every word from a bilingual dictionary.
Then a set of rules disambiguates the forms with lexical or morphological
ambiguities. However, this rule-based lexical disambiguation is very limited,
as it is not feasible to cover all possible contexts for every ambiguous word
with rules.

The rest of the system makes use of a classical transfer procedure. A following module
moves syntactic information between the nodes and the chunks in the tree, and
finally, the tree is reordered according to the basic word order in the target
language. In the last step, the Quechua surface forms are morphologically
generated through a finite state transducer.

\section{CL-WSD with Chipa}
Chipa is a system for cross-language word sense disambiguation (CL-WSD)
\footnote{Chipa the software is named for chipa the snack food, popular in many
parts of South America. It is a cheesy bread made from cassava flour, and
often served in a bagel-like shape in Paraguay. Its Portuguese name,
\emph{p\~{a}o de queijo}, may be familiar to readers. The software is available
at \url{http://redacted}, under the GPL.}. By
CL-WSD, we mean the problem of assigning labels to polysemous words in
source-language text, where each label is a word or phrase type in the target
language.

This framing of word-sense disambiguation neatly addresses the problem of
choosing an appropriate sense inventory, which has historically been a
difficult problem for the practical application of WSD systems.
Here the sense distinctions that the CL-WSD system should learn are exactly
those that are lexicalized in the target language.
The CL-WSD framing also sidesteps the ``knowledge
acquisition bottleneck" \cite{agirre2006word}
hampering other work in WSD. While supervised CL-WSD methods typically require
bitext for training, this is more readily available than the sense-annotated
text that would otherwise be required.

To appreciate the word-sense disambiguation problem embedded in machine
translation, consider for a moment of the different senses of ``have" in
English. In \emph{have a sandwich}, \emph{have a bath}, \emph{have an
argument}, and even \emph{have a good argument}, the meaning of the verb ``to
have" is quite different. It would be surprising if our target language,
especially if it is not closely related, used a light verb that could appear in
all of these contexts.

Concretely considering the Spanish-Quechua translation case, the verb
\emph{tener} (a common translation of \emph{to have}) can often translate into
Quechua as \emph{kay}. But with many objects, such as in \emph{tener hambre} or
\emph{tener miedo} (\emph{to be hungry}, \emph{to be afraid}), we see a single
verb in Quechua, \emph{yarqachikuy} or \emph{manchakuy}, respectively.  Outside
of light verbs, different languages simply make different distinctions about
the world. The Spanish \emph{hermano}, \emph{hijo} and \emph{hija}
(\emph{brother}, \emph{son}, \emph{daughter}) all translate to different
Quechua terms based on the person related to the referent; a daughter relative to
her father is \emph{ususi}, but when described relative to her mother,
\emph{warmi wawa} \cite{academiamayor}.

Chipa, then, must learn to make these distinctions automatically, learning from
examples in available word-aligned bitext corpora. Given such a corpus, we can
discover the different possible translations for each source-language word, and
with supervised learning, how to discriminate between them.  Since instances of
a source-language word may be NULL-aligned, both in the training data and in
actual translations, we allow users to request classifiers that consider NULL
as a valid label for classification, or not, as appropriate for the
application.

The software holds all of the available bitext in a database, retrieving the
relevant training sentences and learning classifiers on demand.
If a source word has been seen with multiple different translations, then a
classifier will be trained for it. If it has been seen aligned to only one
target-language type, then this is simply noted, and if the source word is not
present in the training data, then that word is marked out-of-vocabulary.
Memory permitting, these classifiers and annotations are kept cached for later
usage. Chipa can be run as a server, providing an interface whereby client
programs can request CL-WSD decisions over RPC.

Here classifiers are trained with the scikit-learn machine learning package
\cite{scikit-learn}, using logistic regression (also known as ``maximum
entropy") with the default settings and the regularization constant set to
$C=0.1$. We also use various utility functions from NLTK \cite{nltkbook}. 

For this work, we use familiar features for text classification: the
surrounding lemmas for the current token (three on either side) and the
bag-of-words features for the entire current sentence. We additionally include,
optionally, the Brown cluster labels (see below for an explanation),
both for the immediate surrounding context and the entire sentence.
We suspect that more feature engineering, particularly making use of syntactic
information and surface word forms, will be helpful in the future.

\begin{figure}[t!]
  \begin{itemize}
    \item lemmas from surrounding context (three tokens on either side)
    \item bag of lemmas from the entire sentence
    \item Brown cluster labels from surrounding context
    \item bag of Brown cluster labels from the entire sentence
  \end{itemize}
\caption{Features used in classification}
\label{fig:features}
\end{figure}

\subsection{System Integration}
In order to integrate Chipa into SQUOIA, we added an additional lexical
selection stage to the SQUOIA pipeline, occurring after the rule-based
disambiguation modules. This new module connects to the Chipa server to request
translation suggestions -- possibly several per word, ranked by their
probability estimates -- then looks for words that SQUOIA currently has marked
as ambiguous.

For each word with multiple translation possibilities, we consider each of the
translations known to SQUOIA and take the one ranked most highly in the
results from the classifiers. If there are no such overlapping translations, we
take the default entry suggested by SQUOIA's dictionary.
Notably, since Chipa and SQUOIA do not share the same lexicon and bitext alignments
may be noisy, translations
observed in the bitext may be unknown to the SQUOIA system, and lexical entries in the
SQUOIA dictionary may not be attested in the training data.

\subsection{Learning From Monolingual Data}
While in this work, our target language is under-resourced, we have many
language resources available for the source language. We would like to use these to
make better sense of the input text, giving our classifiers clearer signals for
lexical selection in the target language.

One resource for Spanish is the simple abundance of monolingual text. Given
large amounts of Spanish-language text, we can use unsupervised methods to
discover semantic regularities. In this work we apply Brown clustering
\cite{Brown92class-basedn-gram}, which has been used in a variety of text
classification tasks \cite{turian-ratinov-bengio:2010:ACL} and provides a
straightforward way to add classification features.

The Brown clustering algorithm takes as input unannotated text and produces a
mapping from word types in that text to clusters, such that words in the same
cluster have similar usage patterns according the corpus's bigram statistics.
We can then use this mapping from words to clusters in our classifiers, adding
an additional annotation for each word that allow the classifiers to find
higher-level abstractions than surface-level words or particular lemmas.
The desired number of clusters must be set ahead of time, but is a tunable
parameter.
We use a popular open source implementation of Brown clustering
\footnote{\url{https://github.com/percyliang/brown-cluster}}, described in
\cite{Liang05semi-supervisedlearning}, running on both the Spanish side of our
bitext corpus and on the Europarl corpus \cite{europarl} for Spanish.

\begin{figure*}[t!]
  \begin{tabular}{|l|p{15cm}|}
    \hline
    category  & top twenty word types by frequency \\
    \hline
    countries & francia irlanda alemania grecia italia españa rumanía portugal polonia suecia bulgaria austria finlandia hungría bélgica japón gran\_bretaña dinamarca luxemburgo bosnia \\
    \hline
    more places & kosovo internet bruselas áfrica iraq lisboa chipre afganistán estrasburgo oriente\_próximo copenhague asia chechenia gaza oriente\_medio birmania londres irlanda\_del\_norte berlín barcelona \\
    \hline
    mostly people & hombre periodista jefes\_de\_estado individuo profesor soldado abogado delincuente demócrata dictador iglesia alumno adolescente perro chico economista gato jurista caballero bebé \\
    \hline
    infrastructure & infraestructura vehículo buque servicio\_público cultivo edificio barco negocio motor avión monopolio planta ruta coche libro aparato tren billete actividad\_económica camión \\
    \hline
    common verbs & pagar comprar vender explotar practicar soportar exportar comer consumir suministrar sacrificar fabricar gobernar comercializar cultivar fumar capturar almacenar curar beber \\
    \hline
  \end{tabular}
\caption{Some illustrative clusters found by the Brown clustering algorithm on
the Spanish Europarl data. These are five out of $C=1000$ clusters, and
were picked and labeled arbitrarily by the authors. The words listed are the
top twenty terms from that cluster, by frequency.}
\label{fig:clusters}
\end{figure*}

See Figure \ref{fig:clusters} for some illustrative examples of clusters that
we found in the Spanish Europarl corpus.  Examining the output of the
clustering algorithm, we see some intuitively satisfying results; here we see
clusters corresponding to the names of many countries, some nouns referring to
people, and common verbs. Note that the clustering is unsupervised, and the
labels given are not produced by the algorithm.

\section{Experiments}
Here we report on two basic experimental setups, including an \emph{in-vitro}
evaluation of the CL-WSD classifiers themselves and an \emph{in-vivo}
experiment in which we evaluate the translations produced by the SQUOIA system
with the integrated CL-WSD system.

\subsection{Classification Evaluation}
To evaluate the classifiers in isolation, we produced a small Spanish-Quechua
bitext corpus from a variety of sources, including the Bible, some government
documents such as the constitution of Peru and several short folktales and
works of fiction. The great majority of this text was the Bible.
We used Robert Moore's sentence aligner \cite{DBLP:conf/amta/Moore02}, with the
default settings to get sentence-aligned text.
Initially there were just over 50 thousand sentences; 28,549 were included
after sentence alignment.

\begin{figure*}[t!]
  \begin{center}
  \begin{tabular}{|r|c|c|c|c|c|}
    \hline
    system                    & \multicolumn{5}{|l|}{accuracy} \\
    \hline
    MFS baseline              &  \multicolumn{5}{|l|}{54.54} \\
    chipa, only word features &  \multicolumn{5}{|l|}{65.43} \\
    \hline
           & $C=100$ & $C=200$ & $C=500$ & $C=1000$ & $C=2000$ \\
    \hline
    chipa, +clusters from training bitext &
    66.71 & 67.43 & 68.41 & 69.00 & 69.43 \\
    chipa, +clusters from europarl        &
    66.60 & 67.18 & 67.83 & 68.25 & 68.58 \\
    \hline
  \end{tabular}
  \end{center}
\caption{Results for the \emph{in-vitro} experiment; classification accuracies
over tenfold cross-validation including null-aligned tokens, as percentages. }
\label{fig:theresults1}
\end{figure*}

\begin{figure*}[t!]
  \begin{center}
  \begin{tabular}{|r|c|c|c|c|c|}
    \hline
    system                    & \multicolumn{5}{|l|}{accuracy} \\
    \hline
    MFS baseline              &  \multicolumn{5}{|l|}{53.94} \\
    chipa, only word features &  \multicolumn{5}{|l|}{68.99} \\
    \hline
           & $C=100$ & $C=200$ & $C=500$ & $C=1000$ & $C=2000$ \\
    \hline
    chipa, +clusters from training bitext &
    71.53 & 72.62 & 73.88 & 74.29 & 74.78 \\
    chipa, +clusters from europarl        &
    71.27 & 72.08 & 73.04 & 73.52 & 73.83 \\
    \hline
  \end{tabular}
  \end{center}
\caption{Classification accuracies over tenfold cross-validation, excluding
null-aligned tokens.}
\label{fig:theresults2}
\end{figure*}

During preprocessing, Spanish multi-word expressions identifiable with FreeLing
were replaced with special tokens to mark that particular expression, and both
the Spanish and Quechua text were lemmatized. We then performed word-level
alignments on the remaining sentences with the Berkeley aligner
\cite{denero-klein:2007:ACLMain}, resulting in one-to-many alignments such that
each Spanish word is aligned to zero or more Quechua words, resulting in a
label for every Spanish token.

With this word-aligned bitext, we can then train and evaluate classifiers.
We evaluate here classifiers for the 100 most common Spanish lemmas appearing
in the aligned corpus. For this test, we performed 10-fold cross-validation for
each lemma, retrieving all of the instances of that lemma in the corpus,
extracting the appropriate features, training classifiers, then testing on
that held-out fold.

We report on two different scenarios for the \emph{in-vitro} setting; in one
case, we consider classification problems in which the word in question may be
aligned to NULL, and in the other setting, we exclude NULL alignments. While
the former case will be relevant for other translation systems, in the
architecture of SQUOIA, lexical selection modules may not make the decision to
drop a word. In both cases, we show the average classification accuracy across
all words and folds, weighted by the size of each test set.

Here we compare the trained classifiers against the ``most-frequent sense"
(MFS) baseline, which in this setting is the most common translation for a
given lemma, as observed in the training data.

We additionally show the effects on classification accuracy of adding features
derived from Brown clusters, with clusters extracted from both the Europarl
corpus and the Spanish side of our training data. We tried several different
settings for the number of clusters, ranging from $C=100$ to $C=2000$;
finer-grained distinctions through more clusters seem to lead to consistent
improvements in accuracy.  The training algorithm takes time quadratic in the
number of clusters, which becomes prohibitive fairly quickly. On a modern Linux
workstation, clustering Europarl into 2000 clusters took roughly a day.

The classifiers using clusters extracted from the Spanish side of our bitext
consistently outperformed those learned from the Brown corpus. We had an
intuition that the much larger corpus (nearly two million sentences) would
help, but the clusters learned in-domain, largely from the Bible, reflect
usage distinctions in that domain. Here we are in fact cheating slightly, as
information from the complete corpus is used to classify parts of that corpus.

For summarized results of these first two experiments, please see Figures
\ref{fig:theresults1} and \ref{fig:theresults2}.

\subsection{Translation Evaluation}
In order to evaluate the effect of Chipa on lexical selection in a live
translation task, we used SQUOIA to translate two Spanish passages for which we
had reference Quechua translations. The first is simply a thousand sentences
from the Bible; the second is adapted from the Peruvian government's public
advocacy website \footnote{\emph{Defensoría del Pueblo},
\url{http://www.defensoria.gob.pe/quechua.php}}, which is bilingual and
presumably contains native-quality Quechua. We collected and hand-aligned
thirty-five sentences from this site.

Having prepared sentence-aligned and segmented bitexts for the evaluation,
we then translated the Spanish side with SQUOIA, with various CL-WSD settings
to produce Quechua text. In comparing the output Quechua with the reference
translations, BLEU scores were quite low. The output often contained no 4-grams
that matched with the reference translations, resulting in a geometric mean of
0. So here we report on the unigram-BLEU scores, which reflect some small
improvements in lexical choice.
See Figure \ref{fig:translatioresults} for the numerical results.

\begin{figure*}[t!]
  \begin{center}
  \begin{tabular}{|r|c|c|}
    \hline
    system                           & web test set & bible test set  \\
    \hline
    squoia without CL-WSD            & 28.1         & 24.2            \\
    squoia+chipa, only word features & 28.1         & 24.5            \\
    squoia+chipa, +europarl clusters & 28.1         & 24.5            \\
    squoia+chipa, +bible    clusters & 28.1         & 24.5            \\
    \hline
  \end{tabular}
  \end{center}
  \caption{BLEU-1 scores (modified unigram precision) for the various CL-WSD
  settings of SQUOIA on the two different Spanish-Quechua test sets.}
\label{fig:translatioresults}
\end{figure*}

On the web test set, unfortunately very few of the Spanish words used were both
considered ambiguous by SQUOIA's lexicon and attested in our training corpus.
Enabling Chipa during translation, classifiers are only called on six of the
thirty-five sentences, and then the classifiers only disagree with the default
entry from the lexicon in one case.

We do see a slight improvement in lexical selection when enabling Chipa on the
Bible test set; the three feature settings listed actually all produce
different translation output, but they are of equal quality. Here the in-domain
training data allowed the classifiers to be used more often; 736 of the
thousand sentences were influenced by the classifiers in this test set.

\section{Related Work}
Framing the resolution of lexical ambiguities in machine translation
as an explicit classification
task has a long history, dating back at least to early SMT work at IBM
\cite{Brown91word-sensedisambiguation}.  More recently, Carpuat and Wu have
shown how to use classifiers to improve modern phrase-based SMT systems
\cite{carpuatpsd}.
CL-WSD has received enough attention to warrant shared tasks at recent SemEval
workshops; the most recent running of the task is described in \cite{task10}.
In this task, participants are asked to translate twenty different polysemous
English nouns into five different European languages, in a variety of contexts.

Lefever \emph{et al.}, in work on the ParaSense system
\shortcite{lefever-hoste-decock:2011:ACL-HLT2011}, produced top results for
this task with classifiers trained on local contextual features, with the 
addition of a bag-of-words model of the translation of the complete source
sentence into other (neither the source nor the target) languages. At training
time, the translations of the sentence are extracted from available parallel
corpora, but at training time, the translations must be produced by a
third-party MT system.
This work has not yet, to our knowledge, been integrated into an MT system
on its own.

Rudnick \emph{et al.} \shortcite{rudnick-liu-gasser:2013:SemEval-2013}
prototyped a system that addresses some of the issues with ParaSense, requiring
more modest software infrastructure for feature extraction while still allowing
CL-WSD systems to make use of several mutually parallel bitexts that share a
source language.

There has been some work on CL-WSD for translating into indigenous American
languages; an earlier system for Spanish-Guarani made use of sequence models to
jointly predict all of the translations for a sentence at once
\cite{rudnick-gasser:2013:HyTra}.

Francis Tyers, in his dissertation work \shortcite{tyers-dissertation},
provides an overview of lexical selection systems and describes methods for
learning lexical selection rules based on available parallel corpora. These
rules make reference to the lexical items and parts of speech surrounding the
word to be translated. Once learned, these rules are intended to be
understandable and modifiable by human language experts. For practical use in
the Apertium machine translation system, they are compiled to finite-state
transducers.

Rios and G\"{o}hring \shortcite{riosgonzales-gohring:2013:HyTra} describe
earlier work on extending the SQUOIA MT system with machine learning modules.
They used classifiers to predict the target forms of verbs in cases where the
system's hand-crafted rules cannot make a decision based on the current
context.

\section{Conclusions and Future Work}
We have described the Chipa CL-WSD system and its integration into SQUOIA,
a machine translation system for Spanish-Quechua.
Until this work, SQUOIA's lexical choices were based on a small number of
hand-written lexical selection rules, or the default entries in a bilingual
dictionary. 

We have provided a means by which the system can make some use of
the available training data, both bilingual and monolingual, with very few
changes to SQUOIA itself. We have also shown how Brown clusters, either when
learned from a large out-of-domain corpus or from a smaller in-domain corpus,
provide useful features for a CL-WSD task.

In order make better use of the suggestions from the CL-WSD module, we may
need to expand the lexicon used by the translation system, so that mismatches
between the vocabulary of the available bitext, the translation system itself,
and the input source text do not hamper our efforts at improved lexical
selection. Finding more and larger sources of bitext for this language pair
would of course help immensely.

We would like to learn from the large amount of monolingual Spanish text
available; while the Europarl corpus is nontrivial, there are much larger
sources of Spanish text, such as the Spanish-language Wikipedia.  We plan
to apply more clustering approaches and other word-sense discrimination
techniques to these resources, which will hopefully further improve CL-WSD
across broader domains.

Better feature engineering outside of unsupervised clusters may also be useful.
In the future we we will extract features from the already-available POS tags
and the syntactic structure of the input sentence.

We also plan to apply the Chipa system to other machine translation systems and
other language pairs, especially Spanish-Guarani, another important language
pair for South America.

\bibliographystyle{lrec2006}
\bibliography{saltmil2014}

\end{document}

